\thispagestyle{empty}
\chapter*{Kurzfassung}

Die Bewegungsplanung f{\"u}r Laufroboter ist eine anspruchsvolle Aufgabe und bleibt ein aktives Forschungsgebiet. Besondere Schwierigkeiten ergeben sich aus der effektiven Unteraktuierung, der Komplexit{\"a}t der Mechanismen sowie der nichtlinearen und hybriden Dynamik. Ein verbreiteter Ansatz besteht darin, das Problem in kleinere Teilprobleme zu zerlegen, die nacheinander gel{\"o}st werden. Neue Forschungsergebnisse zeigen, dass Algorithmen aus dem Bereich der optimalen Regelung, insbesondere der Differenziellen Dynamischen Programmierung (DDP), effizientere Bewegungen mit geringeren Kr{\"a}ften und St{\"o}{\ss}en erzeugen.

Diese Masterarbeit leistet einen Beitrag in diese Richtung, indem eine DDP-basierte Ganzk{\"o}rper-Trajektorienoptimierung angewendet, bewertet und erweitert wird, wobei drei Ziele verfolgt werden. Erstens entwickeln wir eine Methode zur Beschr{\"a}nkung von DDP-artigen Algorithmen, um stabile Bewegungspl{\"a}ne zu erzeugen. Zweitens wird der vorgeschlagene Ansatz zur Bewegungsplanung f{\"u}r quasi-statische und dynamische Bewegungen in einer physikalischen Echtzeitsimulation und in realen Experimenten mit dem leichten und biologisch inspirierten humanoiden Roboter RH5 evaluiert. Schlie{\ss}lich werden die Grenzen des Ansatzes und des Systemdesigns anhand hochdynamischer Bewegungen untersucht.

\vfill
\noindent\textbf{Stichworte:} Bewegungsplanung, Differenzielle Dynamische Programmierung, Dynamisches Bipedales Laufen, Ganzk{\"o}rper-Trajektorienoptimierung, Humanoide Roboter, Optimale Regelung
