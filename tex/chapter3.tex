%-----------------------------------------------------------------------------%
%                                                                             %
%    K A P I T E L   3                                                      %
%                                                                             %
%-----------------------------------------------------------------------------%

\chapter{Contact Stability Constrained DDP}\label{c3}
This chapter presents a generic method for integrating contact stability constraints into DDP-like solvers. The key idea is to define inequality constraints for unilaterality, friction and the \gls{CoP} of each contact surface with the goal of generating inherently balanced motions.

\section{The Idea}
In \cref{sec:TheoryStability} we have explored two different criteria for ensuring contact stability for dynamic systems, namely the \gls{ZMP} and \gls{CoP}. As outlined, the application of the \gls{ZMP} is limited due to the assumptions of sufficiently high friction and the existence of one planar contact surface. Since we want to provide a \textit{generic} method that can also be used for e.g. walking up stairs, these simplifying assumptions do not hold anymore. 

Consequently, we decide to model a 6d surface contact, as introduced in \ref{eqn:EoMLeggedRobotSurfaceContact} with dedicated constraints for (i) unilaterality of the contact forces (ii) Coloumb friction on the resultant force, and (iii) \gls{CoP} inside the support area. This approach can be compared to the concept of contact wrench cone \cite{caron2015stability}, without additionally enforcing the yaw torque constraint. These inequality constraints for surface contacts can compactly be summarized as
\begin{subequations}\label{eqn:contractWrenchConeReduced}
\begin{align}
f_i^z &> 0 \label{subeqn:stabilityUnilaterality},\\
\mid f_i^x\mid &\leq \mu f_i^z \label{subeqn:stabilityFrictionX},\\
\mid f_i^y\mid &\leq \mu f_i^z \label{subeqn:stabilityFrictionY},\\
\mid X\mid & \geq C_x \label{subeqn:stabilityCoPPitch},\\
\mid Y\mid & \geq C_y \label{subeqn:stabilityCoPRoll}.
\end{align}
\end{subequations}
%Original CoP constraints from Caron paper for horizontal floor
%\mid \tau_i^x\mid & \leq Yf_i^z \label{subeqn:stabilityCoPPitch},\\
%\mid \tau_i^y\mid & \leq Xf_i^z \label{subeqn:stabilityCoPRoll}.

Let us now detail each line of the approach. 
The first inequality \cref{subeqn:stabilityUnilaterality} accounts for the unilaterality of the contact force. By nature, contact forces always have to be positive since the robot can only \textit{push} from the ground, not \textit{pull} to the ground (\cref{img:simple_contact}). 
Inequality \crefrange{subeqn:stabilityFrictionX}{subeqn:stabilityFrictionY} corresponds to the Coloumb friction, where $\mu$ denotes the static coefficient of friction. From a modeling perspective, this can be interpreted via the concept of spatial friction cones \cite{kao2016contact}. If, and only if the distributed contact forces lie inside their respective friction cones, these constraints are satisfied. 
Finally, inequalitiy \crefrange{subeqn:stabilityCoPPitch}{subeqn:stabilityCoPRoll} constrain the \gls{CoP} to lie inside the rectangular contact area of each foot (see \cref{img:contact_surface}). $C_x$ and $C_y$ denote the x and y position of $\bp_{CoP}$, respectively. These \gls{CoP} constraints prevent the robot from tilting about the edges of the rectangular surface contact. In particular, \cref{subeqn:stabilityCoPPitch} corresponds to a constraint of tilting around the pitch axis and \cref{subeqn:stabilityCoPRoll} prevents tilting around the roll axis.

Both, the unilaterality of the contact forces and the friction cone constraints, are already implemented inside Crocoddyl. However, the central component, bounding the \gls{CoP} to lie inside the support area of each contact foot, is missing. Therefore, the rest of this chapter deals with the derivation of a set of implementable \gls{CoP} constraints and describes the integration of these constraints as a cost function into the Crocoddyl framework.
\begin{figure}[t]
	\begin{subfigure}{.5\textwidth}
		\centering
		\includegraphics[width=.95\linewidth]{img/simple_contact}
		\caption{Body on a horizontal surface.}
		\label{img:simple_contact}
	\end{subfigure}%
	\begin{subfigure}{.5\textwidth}
		\centering
		\includegraphics[width=.77\linewidth]{img/contact_surface}
		\caption{Notations used for \gls{CoP} definition.}
		\label{img:contact_surface}
	\end{subfigure}
	\caption[Contact in the surface plane.]{Contact in the surface plane \cite{caron2015stability}.}
	\label{fig:natural2robot}
\end{figure}


\section{Center of Pressure (CoP) Constraints}
In this section we will derive a universal set of implementable constraints that bound the \gls{CoP} to lie inside the rectangular contact area of each foot.

\subsection{CoP Stability Conditions}
Recapitulate the constraints from inequality \crefrange{subeqn:stabilityCoPPitch}{subeqn:stabilityCoPRoll}. Instead of using the absolute value of $X$ and $Y$, one can also formulate the constraints as
\begin{align}
\begin{split}
\textbf{Pitch}:\quad& -X \leq C_x \leq X,\\
\textbf{Roll}:\quad& -Y \leq C_y \leq Y.
\end{split}
\end{align}
Based on this formulation it becomes evident that the \gls{CoP} is constraint to lie inside the foot geometry visualized in \cref{img:contact_surface}. In fact, these conditions can be represented via four single inequality equations as
\begin{align}\label{eqn:CoPInequalities}
\begin{split}
X + C_x \geq 0, \\
X - C_x \geq 0, \\
Y + C_y \geq 0, \\
Y - C_y \geq 0. \\
\end{split}
\end{align}
These four inequality equations will be used in the following to formulate the \gls{CoP} constraints.   

\subsection{CoP Computation}
Our goal is to determine explicit expressions for $C_x$ and $C_y$ for arbitrary floor orientations, including inclined ground. To this end, consider the computation routine for the \gls{CoP} from \cref{eqn:CoPComputation}
\begin{equation*} 
\bp_{CoP}=\dfrac{\bn\times\myM{\btau_O^c}}{\bfun^c\cdot\bn}.
\end{equation*}
For arbitrary orientations of the contact normal vector $\bn$, we obtain
\begin{equation}\label{eqn:CoPComputationDetailed}
\bp_{CoP}=\dfrac{\bn\times\myM{\btau_O^c}}{\bfun^c\cdot\bn} = \dfrac{\begin{bmatrix} n_x \\ n_y \\ n_z \end{bmatrix} \times \begin{bmatrix} t_x \\ t_y \\ t_z \end{bmatrix}}{\begin{bmatrix} f_x \\ f_y \\ f_z \end{bmatrix} \cdot \begin{bmatrix} n_x \\ n_y \\ n_z \end{bmatrix}} = 
\begin{bmatrix} n_yt_z - n_zt_y \\ n_zt_x-n_xt_z \\ n_xt_y-n_yt_x \end{bmatrix}\cdot \dfrac{1}{f_xn_x+f_yn_y+f_zn_y},
\end{equation}
and solve for the desired position $C_x$ and $C_y$ of the \gls{CoP} as
\begin{subequations}
\begin{align}
C_x&=\dfrac{n_yt_z - n_zt_y}{f_xn_x+f_yn_y+f_zn_y}, \label{subeqn:Cx}\\
C_y&=\dfrac{n_zt_x-n_xt_z}{f_xn_x+f_yn_y+f_zn_y} \label{subeqn:Cy}.
\end{align}
\end{subequations}

\subsection{CoP Inequality Constraints}
Now that we have found explicit expressions for computing the \gls{CoP} (\crefrange{subeqn:Cx}{subeqn:Cy}), we can insert them into \cref{eqn:CoPInequalities}, which gives a set of four \gls{CoP} constraints as 
\begin{align}\label{eqn:CoPInequalityEqs}
\begin{split}
X + \dfrac{n_yt_z - n_zt_y}{f_xn_x+f_yn_y+f_zn_y} \geq 0, \\
X - \dfrac{n_yt_z - n_zt_y}{f_xn_x+f_yn_y+f_zn_y} \geq 0, \\
Y + \dfrac{n_zt_x-n_xt_z}{f_xn_x+f_yn_y+f_zn_y} \geq 0, \\
Y - \dfrac{n_zt_x-n_xt_z}{f_xn_x+f_yn_y+f_zn_y} \geq 0. \\
\end{split}
\end{align}
These conditions can be written in matrix form as:  
\begin{equation}\label{eqn:CoPInequalityMatrix}
\begin{bmatrix}  
Xn_0 & Xn_1 & Xn_2 & 0 & -n_2 & n_1 \\
Xn_0 & Xn_1 & Xn_2 & 0 & n_2 & -n_1 \\
Yn_0 & Yn_1 & Yn_2 & n_2 & 0 & -n_0 \\
Yn_0 & Yn_1 & Yn_2 & -n_2 & 0 & n_0 \\ \end{bmatrix} \cdot
\begin{bmatrix} f^x \\ f^y \\ f^z \\ \tau^x \\ \tau^y \\ \tau^z \end{bmatrix} \geq
\begin{bmatrix} 0 \\ 0 \\ 0 \\ 0 \end{bmatrix},
\end{equation}
and finally yield an implementable set of inquality equations for constraining the \gls{CoP} to lie inside the rectangular contact area of each foot. 


\section{Integration into the Crocoddyl Framework}
This section presents the integration of the derived CoP inequality constraints from \cref{eqn:CoPInequalityMatrix} into the Crocoddyl framework. 

\subsection{Inequality Constraints by Penalization}
In \cref{sec:TheoryConstrainedDDP} we have discussed possible ways of incorporating inequality constraints into DDP-like solvers. Crocoddyl handles inequality constraints, such as joint limits or friction cone, via penalization. 
In numerical optimization, the goal is to minimize a given cost function. In Crocoddyl, an \textit{action model} combines dynamics and cost model for each knot of the discretized \gls{OC} problem from \cref{eqn:totalCost}. The cost function for an action model at knot $n$ can be written as: 
\begin{equation}\label{eqn:costSum}
l_n=\sum_{c=1}^{C}\alpha_c\Phi_c(\bq,\dot{\bq}, \btau), 
\end{equation}
where $C$ different costs $\Phi_c$ are weighted by a respective coefficient $\alpha_c\in R$. The goal of the following two parts is to demonstrate how the inequality \cref{eqn:CoPInequalityMatrix} is implemented inside a novel cost function into the framework.

\subsection{Computation of the Residual}
In numerical analysis, the term \textit{residual} corresponds to the error of an result \cite{shewchuk1994introduction}. For the sake of compactness, we abbreviate \cref{eqn:CoPInequalityMatrix} as 
\begin{equation} \label{eqn:costPositive}
\myM{A} \cdot \myM{w} \geq \myM{0},
\end{equation}
where $\myM{A}$ corresponds to a matrix of \gls{CoP} inequality constraints and $\myM{w}$ is the contact wrench acting on the according foot. The residual $\myM{r}\in \myM{R}_{4\times 1}$ of the cost is retrieved by a simple matrix-vector multiplication:
\begin{equation}\label{eqn:costResidual}
\myM{r} = \myM{A} \cdot \myM{w}.
\end{equation} 

\subsection{Computation of the Cost}
The residual vector depicted in \cref{eqn:costResidual} typically contains non-zero numbers. The resulting scalar \gls{CoP} cost value $\Phi_{CoP}$ is computed via a bounded quadratic activation as
\begin{align}\label{eqn:CoPCostComputation}
\begin{split}
\Phi_{CoP} = 0.5 \cdot ||\myM{r}||^2 \quad &\mid lb >= \myM{r} >= ub \\
\Phi_{CoP} = 0 \quad &\mid lb < \myM{r} < ub.
\end{split}
\end{align}
In order to account for the positiveness of $\myM{r}$ (see \cref{eqn:costPositive}), the bounds are set to $lb=\myM{0}$ and $ub=\infty$, respectively. Finally, this bounded quadratic activation of the residual vector has the following implications:
\begin{itemize}
\item The \gls{CoP} cost is zero, whenever $\bp_{CoP}$ lies inside or on the border of the foot area spanned by $X$ and $Y$,
\item The \gls{CoP} cost increases in a quadratic manner, when $\bp_{CoP}$ exceeds the foot area spanned by $X$ and $Y$.
\end{itemize}  

\subsection{Basic Usage of the CoP Cost}
The cost function is implemented in C++, but can be accessed via Python bindings for versatile and fast prototyping. In the following, a basic example is provided to demonstrate the interface of the \gls{CoP} cost function to the interested reader.
\begin{verbatim}
# 1. Creating the cost model container
costModel = crocoddyl.CostModelSum(state, actuation.nu)
# 2. Defining the CoP cost
footGeometry = np.array([0.2, 0.08]) # dim [m] of the foot area
CoPCost = crocoddyl.CostModelContactCoPPosition(state, 
crocoddyl.FrameCoPSupport(footId, footGeometry), actuation.nu)
# 3. Adding the CoP cost term with assigned weight to the cost model
costModel.addCost("LF_CoPCost", CoPCost, 1e3)
\end{verbatim}

\subsection{List of Contributions}
The contributions of this thesis to the open-source framework Crocoddyl are summarized in two main pull requests. The first one, \hyperlink{https://github.com/loco-3d/crocoddyl/pull/792}{\#792} contains the basic formulation of the \gls{CoP} cost function for contact dynamics action models. With \hyperlink{https://github.com/loco-3d/crocoddyl/pull/830}{\#830}, an additional version is added the special case of impulse dynamics.
A functional unittest that checks the cost against numerical differentiation can be found in the according directory of \cite{crocoddylweb}.
















%\subsection{Backup: Friction Cone constraints}
%\begin{align}
%\begin{split}
%\mid\mid f^x\mid\mid &\leq \mu f^z \\
%\mid\mid f^y\mid\mid &\leq \mu f^z \\
%f^z &> 0
%\end{split}
%\end{align}
%For the case of four edges of the linear approximation of the friction cone, the equations become:
%\begin{equation}
%\begin{bmatrix} 1 & 0 & -\mu \\
%-1 & 0 & -\mu \\
%0 & 1 & -\mu \\
%0 & -1 & -\mu \\
%0 & 0 & -\mu \\ \end{bmatrix} \cdot
%\begin{bmatrix} f^x \\ f^y \\ f^z \end{bmatrix} \leq
%\begin{bmatrix} 0 \\ 0 \\ 0 \\ 0 \\ 0 \end{bmatrix}
%\end{equation}

%\subsection{CoP Constraints: Horizontal Floor}
%For the special case of horizontal floor, with the according normal vector $\myM{n}=[0,0,1]$, the \gls{CoP} can be computed with the help of \cref{eqn:CoPComputationDetailed} to 
%\begin{equation}
%\myM{p}_{CoP} = \begin{bmatrix} -t_y/f_z \\ t_x/f_z \\ 0 \end{bmatrix},
%\end{equation}
%which in turn can be represented by four inequality conditions:
%\begin{align}
%\begin{split}
%X-\dfrac{\tau_y}{f_z} &\geq 0, \\
%X+\dfrac{\tau_y}{f_z} &\geq 0, \\
%Y+\dfrac{\tau_x}{f_z} &\geq 0, \\
%Y-\dfrac{\tau_x}{f_z} &\geq 0. \\
%\end{split}
%\end{align}
%These conditions can be transformed into matrix form for the purpose of implementation as
%\begin{equation}
%\begin{bmatrix} 
%0 & 0 & X & 0 & -1 & 0 \\
%0 & 0 & X & 0 & 1 & 0 \\
%0 & 0 & Y & 1 & 0 & 0 \\
%0 & 0 & Y & -1 & 0 & 0 \end{bmatrix} \cdot
%\begin{bmatrix} f^x \\ f^y \\ f^z \\ \tau^x \\ \tau^y \\ \tau^z \end{bmatrix} \geq
%\begin{bmatrix} 0 \\ 0 \\ 0 \\ 0 \end{bmatrix}.
%\end{equation}




